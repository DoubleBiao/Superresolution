\documentclass{article}

\usepackage{algorithm}
\usepackage{algorithmic}

\usepackage{amsmath}

\begin{document}
\section{Persudo Code of Superresolution}
\begin{algorithm}[H]
	\caption{Dictionary-Learning Process} 
	\label{alg1}
	\begin{algorithmic}
            \STATE \textbf{STEP  1.} Load training images.
            
            \STATE \textbf{STEP  2.} Converts images from RGB color space to YCbCr color space and save the illuminance(Y) value.
            
            \STATE \textbf{STEP  3.} Crop the images and save as \textbf{high resolution images} set.
            
            \STATE \textbf{STEP  4.} Downsampling the high resolution images to get \textbf{low resolution images} set.
            
            \STATE \textbf{STEP  5.} Upsampling the low resolution images to get \textbf{middle resolution images} set.
            
            \STATE \textbf{STEP  6.} Call \textit{collect} function to extract features for each image in middle resolution set and get \textbf{features} matrix.
            
            \STATE \textbf{STEP  7.} Upsampling the low resolution images to get  \textbf{interpolated images} set.
            
            \STATE \textbf{STEP  8.} Subtract each images in high resolution set and in interpolated images set to get patches set.
            
            \STATE \textbf{STEP  9.} Call \textit{collect} function to extract features for each image in patches set and get \textbf{patches} matrix.
            
            \STATE \textbf{STEP 10.} Implement dimensionality reduction on features matrix based on PCA and get \textbf{features\_pca} matrix.
            
            \STATE \textbf{STEP 11.} Call \textit{ksvd} function using features\_pca to train for the \textbf{low resolution dictionary} as well as \textbf{gamma} matrix.
            
            \STATE \textbf{STEP 12.} Calculate high resolution dictionary $D_h$ using formula : $D_h = PQ^T(QQ^T)^{-1}$  (P: patches matrix  Q: gamma matrix)

	\end{algorithmic}
\end{algorithm}

\end{document}